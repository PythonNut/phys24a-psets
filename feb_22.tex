\documentclass[12pt,letterpaper]{hmcpset}
\usepackage[margin=1in]{geometry}
\usepackage{graphicx}
\usepackage{amsmath,mathtools}
\usepackage{tikz}
\usepackage{pdfpages}
\usepackage{xcolor}

% info for header block in upper right hand corner
\name{ }
\class{Physics 24a}
\assignment{Momentum}
\duedate{Monday, February 22, 2016}

\newcommand{\dg}[1]{#1^o}
\newcommand{\m}[1]{\begin{bmatrix} #1 \end{bmatrix}}
\newcommand{\am}[2]{\left[ \begin{matrix} #1 \end{matrix} \right. \left| \begin{matrix} #2 \end{matrix} \right]}

% set numbering style for enumerated lists to be of form (a), (b), (c), etc.
\renewcommand{\labelenumi}{{(\alph{enumi})}}

% command to make 2.5in wide image centered on page
\newcommand{\diagram}[1]{\begin{center}\includegraphics[width=2.5in]{#1}\end{center}}

\begin{document}

\problemlist{4.\{13,15,21,23,24\}}

\begin{problem}[1 - Sand sprayer - KK 4.13]
    A sand-spraying locomotive sprays sand 
    horizontally into a freight car as shown 
    in the sketch. The locomotive and freight
    car are not attached. The engineer in the
    locomotive maintains his speed so that the
    distance to the freight car is constant. 
    The sand is transferred at a rate $dm/dt =
    10\text{kg/s}$ with a velocity $5\text{m/s}$
    relative to the locomotive. The freight car
    starts from rest with an initial mass of 
    $2000\text{kg}$. Find its speed after 
    $100\text{s}$.

    \diagram{img/4_13.png}

\end{problem}

\begin{solution}
    \vfill
\end{solution}
\clearpage


\begin{problem}[2 - Women and flatcar - KK 4.15]
    $N$ women, each with mass $m$, stand 
    on a railway flatcar of mass $M$. They
    jump off one end of the flatcar with 
    speed $u$ relative to the car. The car
    rolls in the opposite direction without
    friction.

    \begin{enumerate}
    \item What is the final velocity of the
        flatcar if all the women jump off 
        at the same time? 
    \item What is the final velocity of the
        flatcar if they jump off one at a time?
        (The answer can be left in the form 
        of a sum of terms.)
    \item Does case (a) or case (b) yield the
        larger final velocity of the flatcar? 
        Can you give a simple physical
        explanation for your answer? 
    \end{enumerate}
\end{problem}

\begin{solution}
    \vfill
\end{solution}
\clearpage


\begin{problem}[3 - Force on a fire truck - KK 4.21]
    A fire truck pumps a stream of water 
    on a burning building at a rate $K$ (kg/s). 
    The stream leaves the truck at angle $\theta$
    with respect to the horizontal and strikes 
    the building horizontally at height $h$ 
    above the nozzle, as shown. What is the 
    magnitude and direction of the force on 
    the truck due to the ejection of the water 
    stream?

    \diagram{img/4_21.png}
\end{problem}

\begin{solution}
    \vfill
\end{solution}
\clearpage

\begin{problem}[4 - Suspended garbage can* - KK 4.23]
    An inverted garbage can of weight $W$ is 
    suspended in air by water from a geyser. 
    The water shoots up from the ground with 
    a speed $v_{0}$, at a constant rate $K$ 
    (mass/time). The problem is to find the 
    maximum height at which the garbage can 
    rides. Neglect any effect of the falling
    water after it rebounds elastically from 
    the garbage can.

    \diagram{img/4_23.png}
\end{problem}

\begin{solution}
    \vfill
\end{solution}
\clearpage


\begin{problem}[5 - Growing raindrop - KK 4.24]
    A raindrop of initial mass $M_{0}$ starts
    falling from rest under the influence of 
    gravity. Assume that the drop gains mass 
    from the cloud at a rate proportional to 
    the product of its instantaneous mass and
    its instantaneous velocity:
    \[
        \frac{dM}{dt} = k M V
    \]
    where $k$ is a constant.

    Show that the speed of the drop eventually
    becomes effectively constant, and give an 
    expression for the terminal speed. Neglect
    air resistance.
\end{problem}

\begin{solution}
    \vfill
\end{solution}

\end{document}
