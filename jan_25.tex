\documentclass[12pt,letterpaper]{hmcpset}
\usepackage[margin=1in]{geometry}
\usepackage{graphicx}
\usepackage{amsmath}

% info for header block in upper right hand corner
\name{---}
\class{Physics 24a}
\assignment{Polar Coordinate Systems}
\duedate{Monday, January 25, 2016}

\newcommand{\pn}[1]{\left( #1 \right)}
\newcommand{\abs}[1]{\left| #1 \right|}
\newcommand{\bk}[1]{\left[ #1 \right]}
\newcommand{\vc}[1]{\left\langle #1 \right\rangle}
\newcommand{\pder}[2]{\frac{\partial #1}{\partial #2}}

% set numbering style for enumerated lists to be of form (a), (b), (c), etc.
\renewcommand{\labelenumi}{{(\alph{enumi})}}

% command to make 2in wide image centered on page
\newcommand{\diagram}[1]{\begin{center}\includegraphics[width=2in]{#1}\end{center}}

\begin{document}

\problemlist{1.\{17,19,21,24,25,27\}}

% 1.17 %
\begin{problem}[1.17]

	A drum of radius R rolls down a slope without slipping. Its axis has acceleration a parallel to the slope. What is the drum’s angular acceleration $\alpha$?

    \diagram{img/1_17}

\end{problem}

\begin{solution}
\end{solution}
\newpage

% 1.19 %
\begin{problem}[1.19]

	By relative velocity we mean velocity with respect to a specified coordinate system. (The term velocity, alone, is understood to be relative to the observer’s coordinate system.)

	\begin{enumerate}
	\item
		A point is observed to have velocity $v_A$ relative to coordinate system $A$. What is its velocity relative to coordinate system $B$, which is displaced from system $A$ by distance \textbf{R}? ($\textbf{R}$ can change in time.)
	\item
		Particles $a$ and $b$ move in opposite directions around a circle with angular speed $\omega$, as shown. At $t = 0$ they are both at the point $r = l\hat{j}$, where $l$ is the radius of the circle.
		Find the velocity of a relative to b.
	\end{enumerate}

	\diagram{img/1_19}

\end{problem}
\newpage

\begin{solution}
\end{solution}

% 1.21 %
\begin{problem}[1.21]

	A particle moves in a plane with constant radial velocity $\dot{r} = 4$ m/s, starting from the origin. The angular velocity is constant and has magnitude $\dot{\theta} = 2$ rad/s. When the particle is 3 m from the origin, find the magnitude of (a) the velocity and (b) the acceleration.

\end{problem}

\begin{solution}
\end{solution}
\newpage

% 1.24 %
\begin{problem}[1.24]

	A tire of radius $R$ rolls in a straight line without slipping. Its center moves with constant speed $V$. A small pebble lodged in the tread of the tire touches the road at $t = 0$. Find the pebble’s position, velocity, and acceleration as functions of time.

	\diagram{img/1_24}

\end{problem}

\begin{solution}
\end{solution}
\newpage

% 1.25 %
\begin{problem}[1.25]

	A particle moves outward along a spiral. Its trajectory is given by $r = A\theta$, where $A$ is a constant. $A = (1/\pi)$ m/rad. $\theta$ increases in time according to $\theta = \alpha t^2/2$, where $\alpha$ is a constant.

	\begin{enumerate}
	\item
		Sketch the motion, and indicate the approximate velocity and acceleration at a few points.
	\item
		Show that the radial acceleration is zero when $\theta = 1/\sqrt{2}$ rad.
	\item
		At what angles do the radial and tangential accelerations
have equal magnitude?
	\end{enumerate}

\end{problem}

\begin{solution}
\end{solution}
\newpage

% 1.27 %
\begin{problem}[1.27]

	A peaked roof is symmetrical and subtends a right angle, as shown. Standing at a height of distance $h$ below the peak, with what initial speed must a ball be thrown so that it just clears the peak and hits the other side of the roof at the same height?

	\diagram{img/1_27}

\end{problem}

\begin{solution}
\end{solution}

\end{document}
